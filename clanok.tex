% Metódy inžinierskej práce

\documentclass[10pt,oneside,slovak,a4paper]{article}

\usepackage[slovak]{babel}
%\usepackage[T1]{fontenc}
\usepackage[IL2]{fontenc} % lepšia sadzba písmena Ľ než v T1
\usepackage[utf8]{inputenc}
\usepackage{graphicx}
\usepackage{url} % príkaz \url na formátovanie URL
\usepackage{hyperref} % odkazy v texte budú aktívne (pri niektorých triedach dokumentov spôsobuje posun textu)

\usepackage{cite}
%\usepackage{times}
\usepackage{indentfirst}


\title{Knižnica ešportu\thanks{Semestrálny projekt v predmete Metódy inžinierskej práce, ak. rok 2022/23, vedenie: Ing. Vladimír Mlynarovič, PhD.}} % meno a priezvisko vyučujúceho na cvičeniach

\author{Matej Kočík\\[2pt]
	{\small Slovenská technická univerzita v Bratislave}\\
	{\small Fakulta informatiky a informačných technológií}\\
	{\small \texttt{xkocikm@stuba.sk}}
	}

\date{\small 6. november 2022} % upravte



\begin{document}

\maketitle

\begin{abstract}
%\ldots 

E-športy sú novovytvárajúcim sa odvetvím športu. V tejto práci si obzrejmíme históriu e-športu~\ref{historia} , ktorá siaha do blízkej minulosti. Predmetom nášho skúmania bude aj hráčska terminológia~\ref{slovnik}, ktorú zahŕňajú najpoužívanejšie pojmy a otázka, či je e-šport naozaj druhom športu~\ref{esportakosport} . 

Chceli by sme predstaviť slovník hráčskeho slangu. Slová vyskytujúce sa v tomto slovníku samozrejme závisia aj od typu hier~\ref{hry} . To znamená, že zväčša každá hra má osobitné výrazy, ktoré hráči počas jej hrania používajú. 

V neposlednom rade sa budeme venovať i konkrétnym, najpopulárnejším hrám~\ref{hry} , ich pravidlám a najznámejším hráčom~\ref{klasifikacia:rozdelenie} . Na záver sa budeme zaoberať medzinárodnými turnajmi v jednotlivých hrách~\ref{hry} . 
\end{abstract}



\section{Úvod}

%\\

Nová doba, nové informácie, nové inovácie. V technologickom priemysle ku koncu dvadsiateho storočia nastal oborvský ošiaľ v rámci počítačov. Boli spístupnené širokej verejnosti a spočiatku nepredstaviteľná predstava sa transformovala do skutočnosti. V spojení s vynájdením internetu, ľudsko zaznamenalo najväčší technologický pokrok v histórii. Práve to otvorilo pomyselnú bránu neobmedzených možností~\ref{historia} . Ľudia odjakživa fantazírovali a vytvárali si predstavy o nereálnom svete čo je v dnešnom pomímaní zadefinované ako virtuálna realita. Rapídny pokrok spôsobil vytváranie prvotných hier až po také grafické zdokonalenie hier ako ich poznáme dnes. Vývoj hier a gaming samozrejme priniesol svoje negatíva, ale i pozitíva. E-šport sa vytváral ruka v ruke s počítačmi, internetom a všeobecne online svetom. E-športy strhli masy ľudí a stali sa hlavným zdrojom zábavy, oddychu…\cite{a1}


\section{Stručná história elektronického športu} \label{historia}

Prvý turnaj sa podľa dostupných informácií konal na Stamfordskej univerzite. Výhercom Spacewar bolo udelené ročné predplatné časopisu Rolling Stones. Jedným z prvých turnajov, ktoré zaujali nie len obrovské množstvo hráčov, ale aj divákov bol turnaj v hre Space Invaders v roku 1980.

Najväčší ošiaľ so sebou priniesla herná konzola Atari 2600. Významne sa pričinila o presun hráčskeho priestoru z herien do pohodlia domova. Existovala aj zdokonalená verzia tejto konzole, ktorá stála v tom čase približne dve tisíc amerických dolárov. Walter Day založil prvý hráčsky tím v roku 1983. Bol vášnivým priekopníkom hier a pokorovaní rekordov v miestnych herňách a práve toto hobby ho doviedlo vytvoriť Americký národný videoherný tím.  

90. roky sa vyznačovali najmä FPS(First person shooter – pohľad strelca z prvej osoby) hrami. Nosným bodom bolo, že FPS hry sa ovládali pomocou počítača a myši, čo bolo medzi hráčmi veľmi obľúbené. Spomedzi všetkých hier vytŕčali najmä RTS(Real time strategy) hry. Ďalším dôležitým milníkom je vytváranie online serverov, ktoré hráčom umožňovali súťažiť proti ostatným hráčom v globálnom merítku. Kto však prístup k internetu a online serverom nemal, mohol navštíviť internetové kaviarne, ktoré poznáme aj dnes. Veľké množstvo peňazí sa nalialo do e-športu najmä v roku 1997. Výherca turnaju si mohol prilepšiť o čiastky v desať tisícoch dolárov. \cite{a1}


\section{Klasifikácia ešportu} \label{klasifikacia}

\subsection{Definícia ešportu a jeho štruktúra} \label{klasifikacia:definicia}

Ešport alebo elektronický šport je pojem pre súťažné hranie počítačových alebo konzolových hier. Hlavným cieľom ešportu je organizovať súťaže v rôznych hrách ako aj vytváraním organizácií, tímov a zjednocovaním hráčov po celom svete. Spočiatku sa hranie hier – gaming považoval iba za akúsi voľnočasovú aktivitu. Neskôr prišiel pojem ešport, ktorý zmenil pohľad ľudí a spoločnosti na túto oblasť a začal sa klásť do profesionálnej roviny.\cite{a3}

\subsection{Najznámejšie organizácie a tímy} \label{klasifikacia:organizacie}

Hráčske organizácie majú pod sebou profesionálnych hráčov a každá sa samostatne zameriava na určitý typ hry. Ich hlavnou úlohou je pre hráčov vytvoriť čo najpohodlnejšie, najpríjemnejšie a najlepšie prostredie pre rozvoj ich schopností. Podporujú hráčov finančne ako aj po stránke sponzoringu ich herného vybavenia. 

Profesionáli musia mať čo najvýkonnejšie a najrýchlejšie počítače, myšky prispôsobené ich preferencii ako aj kvalitný monitor až po veci, ktoré spadajú do zdravotnej sféry. Poznáme samozrejme aj organizácie jednotlivcov ako napríklad v hre Hearthstone – každý hráč reprezentuje sám seba ako jednu organizáciu. V našich končinách zatiaľ nemáme profesionálny tím, ktorý by našu krajinu mohol reprezentovať na medzinárodnej svetovej úrovni.\cite{a4} Najznámejšími organizáciami sú: FaZe Clan(CS:GO, PUBG, Call of duty, FIFA, Fortnite), OpTic Gaming(Valorant, Call of duty, Overwatch), G2 Esports(Valorant, CS:GO, LOL…), Natus Vincere(CS:GO, FIFA, Fortnite, LOL, Valorant, PUBG).\cite{a5}


\subsection{Rozdelenie hráčov, streameri, influenceri} \label{klasifikacia:rozdelenie}

Každý šport má svoje určité špecifiká, ktoré ich definuje a dáva mu určitú dôležitosť. V klasickom športe poznáme delenie hráčov na amatérskych a profesionálnych. Rozdiel nie je ani v ešporte. Ako súčasť športu sa hráči počítačových hier rovnako rozdeľujú ako pri klasickom športe. Jedinou indifernetnou skupinou sú v ešportoch poloprofesionálni hráči. 
Skupina amatérov a poloprofesionálov za svoje výsledky, výhry nedostáva nijaké peňažné ohodnotenie. Sú to najmä vášnivý hráči, ktorým organizácia zabezpečuje čo najlepšie podmienky pre trénovanie a vybavenie. Je im poskytnutá finančná čiastka na cestovanie na turnaj. 

Profesionáli s organizáciou podpisujú zmluvu. Aké podmienky majú amatéri a poloprofesionáli take isté sú vytvorené pre profesionálov, ale samozrejme vo väčšej miere. Znamená to, že okrem najlepšieho vybavenia majú profesionálni hráči aj finančné ohodnotenie. Profesionáli sú viazaní zmluvou na “full-time job”.\cite{a1}

Množstvo hráčov začalo so streamovaním a následne sa tvrdým tréningom dostali na úroveň profesionálov. Medzi najznámejších slovenských hráčov patrí Ladislav Kovács alias GuardiaN. Je profesionálnym hráčom hry Counter-Strike: Global Offensive(CS:GO)~\ref{hry:csgo} . Túto hru hrával od jej prvej verzie Counter-Strike 1.6, cez neskoršiu verziu Counter-Strike Source, až po aktuálnu verziu Counter-Strike: Global Offensive~\ref{hry:csgo} . Súťažil za najznámejšie svetové tími v CS:GO ako napr. Virtus.pro, Natus Vincere, FaZe clan. Jeho najväčším úspechom bolo umiestnenie na druhom mieste na svete v roku 2015.

Ďalším priekopníkom hry CS:GO je DEV1 vlastným menom Ivan Lazarov. Aktuálne sa už neživí ako profesionálny hráč, ale ako streamer a komentátor turnajov. Je 3-násobným majstrom Českej republiky a jedným z najznámejších osobností herného sveta.\cite{a6}

Kvalitná dobrá propagácia, reclama musí zaujať divákov. Každý streamer a influencer má svoje charakterové črty, ku ktorým diváci vzhliadajú. Sú pre divákov unikátny a zaujímavý. Preto hlavným cieľom organizácií vyhľadať streamerov a influencerov s čo najväčším možným dosahom. Títo známi ľudia propagovajú nejaký turnaj, súťaž, aby bola pre divákov čo najzaujímavejšie vykreslená.\cite{a1}


%\paragraph{Veľmi dôležitá poznámka.}
%Niekedy je potrebné nadpisom označiť odsek. Text pokračuje hneď za nadpisom.


\section{Hráčska terminológia} \label{slovnik}

\section{Najpopulárnejšie hry, najlepší hráči a veľké turnaje} \label{hry} 

\subsection{Counter-Strike: Global Offensive} \label{hry:csgo}

Counter-Strike: Global Offensive je bezpochybne jednou z najznámejších, najpopulárnejších a najhranejších hier na svete. CS:GO je kompetitívna multiplayer FPS hra. Je to tímová hra, ktorá sa hraje štýlovm 5 hráčov(1. tím) proti 5 hráčom(2. tím). Dve predošlé verzie boli Conter-Strike 1.6 a Counter-Strike: Source. Je veľmi obľúbená najmä medzi mladými dospievajúcimi ľuďmi. Pôvodne sa hrala na 16 výherných kôl, čo stále platí pre súťažnú formu, však rokmi aktualizácií a vylepšení priniesli v súčaasnej dobe aj formát Wingman a skrátené kompetitívne zápasy na 9 víťazných kôl. Wingman je kompetitivný zápas, na ktorom sa zúčasťňujú dva tímy s dvoma hráčmi na každej strane. Poznáme dva tábory – tábor teroristov(T site) a tabor counter-teroristov(CT site). Táto hra sa hrá na 16 víťazných kôl. Po uplynutí polovice vypísaných kôl si tímy menia strany.\cite{a1}

Hlavným cieľom teroristického táboru je dostať sa na jedno z dvoch miest na mape, označených písmenami A a B, a následne na jedno z nich položiť, aktivovať bombu. Nastavený časový limit na položenie bomby sú dve minúty. No tu práca teroristov ešte nekončí, práve začína. Najťažšou časťou hry pre teroristov je ubránenie bomby pred CT respektíve udržanie bomby aktívnej pokiaľ nevybuchne. Teroristický arzenál je odlišný od toho Counter-teroristického. V inventári s anachádzajú zbrane ako AK-47, Glock, SG 550… Je dôležité spomenúť, že zbrane sú v reálnom svete naozaj vyrábané. Každý prototyp je teda nadizajzovaný podľa skutočnej predlohy. Hoci je väčšina zbraní odlišných, sú aj take, ktoré sa zhodujú pre oba tímy. Napr. AWP, pistol Desert Eagle alebo IMI Negev.\cite{a1}

Na strane obrancov stoja counter-teroristi. Obidve strany si vyžadujú vytvorenie nejakej stratégie respektíve taktiky na dosiahnutie cieľa a splnenie svojej úlohy. Omylom nie je ak povieme, že si tábor counter-teroristov musí určiť viac premyslenejšiu taktiku. Ich úlohou je správne a efektívne pokryť pozície na miestach s označením A a B na uloženie bomby. Ak sa im nepodarí eliminovať teroristický tím, ktorý aktivuje bombu v tom prípade ju musia CT do časového limitu explózie zneškodniť. Ich inventár zbraní pripomína arzenál ozbrojených síl, špeciálnych bojových jednotiek…\cite{a1} 

Na začiatku každého kola si môže hráč v krátkom čase vybrať nejakú zbraň, čo závisí od jeho finančného stavu(peniaze sa získavajú na základe vyhratého/prehratého kola). Máp v hre CS:GO je mnoho, ale najhranejšími I používanými mapami na turnajoch sú: Dust II(vojnou zničené mesto v juhozápadnej Ázii), Inferno(talianske sedliacke mestečko), Mirage(vojnou zničené mesto), Overpass(budovy, parky, základne), Train(vlakové depo).\cite{a1}


\subsection{League of Legends} \label{hry:lol}

\subsection{FIFA} \label{hry:fifa}

\section{Ešport ako nový druh športu?} \label{esportakosport}

\section{Záver} \label{zaver} % prípadne iný variant názvu


%\acknowledgement{Ak niekomu chcete poďakovať\ldots}


% týmto sa generuje zoznam literatúry z obsahu súboru literatura.bib podľa toho, na čo sa v článku odkazujete
\bibliography{literatura}
\bibliographystyle{plain} % prípadne alpha, abbrv alebo hociktorý iný
\end{document}
